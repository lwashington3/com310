%! Author = Len Washington III
%! Date = 1/18/24
%! compiler = pdflatex

% Preamble
\documentclass[title={}]{com310notes}

% Packages

% Document
\begin{document}

\maketitle

\begin{table}[H]
    \centering
    \begin{threeparttable}
		\caption{}
		\label{tab:}
		\begin{tabular}{|c  c  c  c|}
			\hline
			& \textbf{voicing} & \textbf{place of articulation (poa)} & \textbf{manner}\\
			\hline
			b: & voiced & bilabial & stop\\
			p: & voiceless & bilabial & stop\\
			m: & voiced & bilabial & nasal\\
			f: & voiceless & lab-dental & fricatives\\
			v: & voiced & lab-dental & fricatives\\
			\hline
		\end{tabular}
		\begin{tablenotes}
			\small
			\item
		\end{tablenotes}
	\end{threeparttable}
\end{table}

\newcommand{\stopcon}[1]{\textcolor{red}{#1}}% Stop consonants

\begin{table}[H]
    \centering
    \begin{threeparttable}
		\caption{}
		\label{tab:}
		\rowcolors{2}{black!30}{white}
		\begin{tabular}{|c|c|c|c|c|c|c|c|c|}
			\hline
			& \textbf{lips} & \textbf{lab-dental} & \textbf{dental} & \textbf{alv} & \textbf{post-alv} & \textbf{palatul} & \textbf{velar} & \textbf{glottal}\\
			\hline
			Stops & \stopcon{p} \stopcon{b} & & & \stopcon{t} \stopcon{d} & & & \stopcon{k} \stopcon{g} & \stopcon{uh} \stopcon{oh}\\
			Nasal & \stopcon{m} & & & n & & & sin\underline{g} &\\
			Fricatives & & \stopcon{f} \stopcon{v} & thin/this & s z & sh zh & & &\\
			Affricatives & & & & & ch, dge, J & & &\\
			Approximates & w & & & l r & & j & &\\
			\hline
		\end{tabular}
		\begin{tablenotes}
			\small
			\item alv means alveolar ridge, red highlights are stop consonants.
		\end{tablenotes}
	\end{threeparttable}
\end{table}


\end{document}