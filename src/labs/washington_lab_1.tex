%! Author = Len Washington III
%! Date = 1/15/24

% Preamble
\documentclass[lab=1,title={Speaking rate}]{com310lab}
\usepackage{amsmath}

% Document
\begin{document}

\maketitle

\begin{overview}
	Speaking rate refers to how fast or slow one speaks.
	For example, many people describe the variety of English from the rural American South as exhibiting a ``drawl'' that is characterized by slow speech, among other attributes.
	In contrast, speech from the urban North East is sometimes described as fast and under-articulated.
	In addition, the speech of women is often described as faster, compared to the speech of men.
	This claim about gender differences in speaking rate is seen not only in popular accounts of speech, but also in supposedly academic publications, e.g. Brizendine (2007), The Female Brain.
	(All of these claims, by the way, turn out to be false, but that’s another matter).\\
\end{overview}


\begin{problem}
	Various methods are used to measure speaking rate.
	One method estimates the number of words-per-minute of a speech sample, while another method estimates the number of syllables-per-second.
	They are calculated as follows:\\

	\begin{equation*}
	\begin{aligned}
		&\mbox{Words per minute} = \frac{\mbox{\# words in speech sample}}{\mbox{duration of speech sample, in seconds}} \times 60 \mbox{ seconds}\\
		&\mbox{Syllables per second} = \frac{\mbox{\# syllables in speech sample}}{\mbox{duration of speech sample, in seconds}}
	\end{aligned}
	\end{equation*}\\

	In theory, both methods should yield similar results: speech found to be fast using one method should also be seen as fast using the other.
	Your job is to determine whether this expectation is true and explain why.
\end{problem}\\

\begin{task}
	Record the paragraph below using any voice recorder app.
	For example, ``voice memos'' on an iPhone will work, so long as you are able to find start and end times for any sentence.
	If you prefer, you can also use any of the recorded files in the ``Lab 1 files'' folder from past students who have agreed to let other students use their files.
	\begin{displayquote}
		Today was busy with preparations.
		Many of the students collected several important statements.
		John gave the book to Mark then put the keys on the chair.
		Sally opened the closet and removed a sweater.
		Fred closed the door and left the room.
		Mary delivered the computer to Susan then handed the papers to the teacher.
		All of the class had lots of things to do.
		Eventually, everyone completed the task they needed to perform.
	\end{displayquote}
	Then, compute the speaking rate for each sentence, using both words-per-minute (w.p.s.) and syllables-per-second (s.p.s) for each sentence as described above.
	You need no special tools to count the number of words and syllables in each sentence.
	If you need help with counting syllables, consult any dictionary--wiktionary is fine.
	To measure the duration of the speech sample, open the recording of your file, delete any pauses or false starts, and then select one sentence at a time and note its duration in seconds.
	Now, examine differences across certain sentences.
	For the purposes of this lab, ignore the first and last sentences.
	The other sentences are numbered as follows:
	\begin{displayquote}
		Sentence 1: Many of the students$\dots$\\
		Sentence 2: John gave$\dots$\\
		Sentence 3: Sally opened$\dots$\\
		Sentence 4: Fred closed$\dots$\\
		Sentence 5: Mary delivered$\dots$\\
		Sentence 6: All of$\dots$\\
	\end{displayquote}
	Compare w.p.m.\ and s.p.s.\ for the odd-numbered sentences to the same measures for the even-numbered sentences.
	Generate a table or graph showing your results.
	Summarize the results, and address the problem raised above, namely whether measures of speaking rates are consistent.
	If you find any differences, use the data to offer an explanation.
	(Hint: the measures are unlikely to be consistent with each other).
\end{task}\\

\begin{writeup}
	Follow the format for writing up phonetics lab reports and turn in your report on the day it is due.
\end{writeup}

\end{document}