%! Author = Len Washington III
%! Date = 3/12/24

% Preamble
\documentclass[lab={4},title={Phonetics of toddler speech},addfiller=true,turnin=false]{com310lab}

% Packages

% Document
\begin{document}

\maketitle

\begin{overview}
	Children acquiring their native languages will often infer phonemes and phonological processes that aren't actually attested in the target language.
	Usually, these inferences affect whole classes of segments.
	For example, a child may exhibit devoicing on all word-final segments, they may substitute \ipa[w] for all glides, or they may reduce all consonant clusters to just one, whichever has the most constricted articulation.
	These `errors' tend to show similar patterns across children at the same developmental stage, but they vary according to degree of acquisition.
	That is, children in the latest stages of acquisition exhibit adult-like sound systems, whereas children in the earliest stages exhibit the least adult-like speech.
	Yet, at all stages, a child's phonology is systematic and rule governed.
\end{overview}

\begin{task}
	Retrieve the set of files found in the folder in Blackboard under Lab 4.
	These are recordings of a 2 $\frac{1}{2}$ year-old native speaker of English.\\

	Examine the recordings and state the general principles that appear to be affecting the toddler's speech.
	For example, if \ipa{/m,n,N/} in adult speech surface as \ipa{b,d,g} at the beginning of words in the toddler's speech, state that nasals become stops at the beginning of words.
	Likewise, if the toddler has acquired, for example, positive VOT and aspiration for word initial \ipa{/p,t,k/}, state that the child has fully acquired native production of voiceless stops.
	For each observation, provide evidence for your claim.
	For example, if you observe that \ipa{/l/} is consistently produced as \ipa{/r/}, examine the acoustic characteristics and confirm that F3 for the child's \ipa{/l/} is low, as would be expected of an \ipa{/r/}.
	Account for as many classes of segments as possible: stops, nasals, fricatives, glides, and vowels.
	Also account for consonant voicing, syllable structure, and anything else you observe.
	Be specific in your observations.
	Refer to areas of the relevant spectrograph in support of your hypothesis.
	The facts of the matter are not nearly as important as showing that you can cite a line of evidence in favor or against a particular hypothesis.\\

	Transcription of the files are included in the file names.
	Two long files have the following transcriptions: (1) \textit{Once upon a time there was a little boy with blonde hair and blue eyes.
	That little boy's name was Alex.
	One night that little boy wanted to go to the playground with the cars in there.}
	(2) \textit{I make, I was, I, I wanna, I was making.., I wanna make cookies.
	Do you see my oven mitt?
	I will go, I go to my kitchen, go to ?? go to ?? kitchen, go to that kitchen, you want to make$\dots$kitchen.}\\
\end{task}

\begin{writeup}
	\pagebreak
\end{writeup}

\labtitle

\begin{topic}
	\\
\end{topic}

\begin{issue}
	\\
\end{issue}

\begin{hypothesis}
	\\
\end{hypothesis}

\begin{method}
	\\
\end{method}

\begin{results}
	\\
\end{results}

\begin{discussion}
	\\
\end{discussion}

\end{document}